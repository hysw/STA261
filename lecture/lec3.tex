\subsection{Wednesday, July 8, 2015}


\subsubsection{Confidence interval}

An \kw{interval estimator}(confidence intervals) is a rule specifying the method for using the sample measurements to calculate two numbers that form the endpoints of the interval.

Given $\Pr(\hat{\theta}_L\leq\theta\leq\hat{\theta}_U) = 1-\alpha$, the probability $(1-\alpha)$ is the \kw{confidence coefficient}.

Let $X$  be a random sample from a distribution that depends on a parameter $\theta$.
Let $g(X,\theta)$ be a random variable whose distribution is the same for all  $\theta$.
Then $g$ is called a \kw{pivotal quantity} (or simply a pivot).

8.5, 8.6 and 8.8

\begin{itemize}
\item \todo{pivotal method at p407}
\item Large Sample CI concerning $\mu$
\item Likelihood Inference
\item Likelihood Principle
\item \href{https://en.wikipedia.org/wiki/Sufficient_statistic}{Suffcient Statistics}
\item factorization theorem
\item minimal sufficient statistic
\end{itemize}

Suffcient Statistics is not unique
